\chapter{Revisão Bibliográfica}

Apesar da crescente aplicação de manipuladores robóticos, estes ainda são
normalmente utilizados em ambientes bem estruturados, sem limitações de espaço e
tamanho para construção de uma base pesada e rígida. Por esse motivo, há uma
carência na literatura em relação a consideração de bases pouco rígidas e o efeito
dinâmico causado no robô, salvo alguns casos para aplicações mais específicas,
como robôs espaciais, que abordam este efeito.
Por outro lado, manipuladores de elos e juntas elásticas já são bastante
explorados por diversos autores.
\citet{dwivedy2006dynamic} descrevem os métodos e modelos presentes na
literatura para análise dinâmica de manipuladores flexíveis.

Espaçonaves e satélites buscam minimizar massa e momentos de inércia de seus
componentes, por isso possuem componentes leves e
esbeltos, resultando em partes muito flexíveis. Estes componentes por vezes
necessitam de serviços de reparo, reabastecimento, e aprimoramentos. Para
executar esta tarefa, em órbita, são utilizados manipuladores espaciais, que
capturam e manipulam a espaçonave. A flexibilidade dos componentes
da espaçonave facilmente produz vibrações, que devido a falta de
amortecimento atmosférico no espaço, podem se propagar e fadigar estes
componentes.
\citet{xu2014dynamics} propõem um modelo dinâmico para manipuladores espaciais,
considerando um manipulador rígido entre bases rígidas, mas com apêndices
frágeis e flexíveis, como painéis solares e antenas.
\citet{torres1993path} propõem um planejamento de trajetória para estes
manipuladores que minimizam as vibrações na estrutura de suporte.

Outra abordagem é para sistemas chamados macro/micromanipuladores. Consistem
em robôs de pequeno porte (micro) montados na ponta de outro maior (macro),
geralmente usados para aumentar o alcance do sistema. \citet{book1999inverse}
modelaram a dinâmica desse sistema, para casos de um macromanipulador não
atuado e esbelto, e propuseram um sistema de controle para amortecer as
vibrações do macromanipulador a partir das forças de inércia do micromanipulador.

A modelagem cinemática de manipuladores industriais seguirá a abordagem clássica
para determinação da equação de cinemática direta, pelo método de
\citet{hartenberg1955kinematic}. Para cálculo das funções de cinemática inversa,
há diversos métodos, dentre as opções estão: Jacobiano Transposto,
Pseudoinversa e Mínimos Quadrados Amortecidos (\citet{buss2004introduction}) ou
os subproblemas de Paden-Kahan (\citet{murray1994mathematical}).

\citet{huston1991multibody} apresenta a dinâmica
de sistema multicorpos, tais como mecanismos, veículos, estruturas e robôs, 
modelados como sistemas mecânicos e estruturais e
\citet{shabana2013dynamics} sua modelagem por componentes rígidos e
deformáveis. A dinâmica desses sistemas é não-linear, apresentando problemas complexos que, na maioria dos casos, só pode ser resolvido numericamente.

O modelo dinâmico tem como objetivo encontrar as equações de movimento.
Uma alternativa aos métodos clássicos de Newton-Euler e D'Alembert encontra-se
no Método de Kane (\citet{kane1985dynamics}). A sistematização deste método para sistemas mecânicos multicorpos, não lineares, é apresentada por
\citet{lesser1995analysis}.




