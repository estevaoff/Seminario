\chapter{Introdução}

A aplicação da tecnologia de manipuladores robóticos se dá principalmente
em ambientes de automação industrial. Porém, cada vez mais robôs estão
substituindo humanos em serviços fora de um ambiente industrial e
estruturado\footnote{Ambiente estruturado: é o ambiente onde os parâmetros
necessários à operacionalidade do sistema robótico podem ser identificados e
quantificados.}. Quando a aplicação se dá em um ambiente de automação
industrial, este é chamado de robô industrial e quando se dá fora, ou
\textit{in situ}, este é chamado robô de serviço \cite{ISO8373}.

Apesar da grande variedade disponível hoje no mercado, os manipuladores
industriais aplicados em operações \textit{in situ} ainda são um desafio.
Um dos principais motivos é que os sistemas de controle destes
manipuladores são projetados para atuar em ambientes bem estruturados, e não
controlam efeitos externos causados pelo ambiente. Isto porque as
estruturas de controle clássicas são capazes apenas de medir e corrigir os
parâmteros associados ao sistema de coordenadas de referência do próprio
manipulador.

No entanto, ao executar uma determinada tarefa, os torques que atuam nas
juntas do manipulador resultam em esforços em sua base, que por
sua vez, pode ser pouco rígida. Neste caso, se a estrutura da base se deforma
dinamicamente, não é possível garantir que os requisitos da tarefa sejam
cumpridos. Parâmetros como velocidade, posição e orientação do manipulador serão
alterados devido a flexibilidade da base.
Portanto, considerar que o robô está instalado sobre uma base perfeitamente
rígida pode não ser adequado em muitos casos de aplicações \textit{in situ}.

Se não é possível quantificar o comportamento dinâmico da base, não garante-se
a premissa de ambiente bem estruturado, requisito da maioria dos manipuladores
industriais, e nos casos de grande flexibilidade o método de controle original
pode ser insuficiente.
Logo, a necessidade de se conhecer e quantificar o comportamento dinâmico da
base onde será instalado um manipulador robótico pode ser crucial para garantir
a viabilidade do serviço. 

O projeto EMMA -- Metodologia de revestimento robótico de turbinas \textit{in
situ}, tem como objetivo desenvolver um sistema robótico para realizar o
revestimento por aspersão térmica em pás de turbinas hidráulicas,
dentro do ambiente da turbina, i.e., aplicação de revestimento a uma pá
instalada, reduzindo significativamente o tempo de parada da turbina
~\citep{Freitas2017}.
Este projeto é uma parceria da empresa Energia Sustentável do Brasil (ESBR),
Laboratório de Controle e Automação, Engenharia de Aplicação e Desenvolvimento
(LEAD/PEE/COPPE) e a empresa THIRTEEN ROBOTICS. O projeto tem o principal
desafio de se transportar um sistema robótico completo para o interior do
ambiente confinado de uma turbina do tipo Kaplan, que tem acesso limitado por
uma escotilha de $800~mm$ de diâmetro. Logo, a solução para a base do
manipulador deve ser leve e modular para permitir a entrada, transporte e
montagem no interior.

O sistema de revestimento por asperção térmica HVOF (\textit{High Velocity
Oxygen Fuel}) requer os seguintes parâmetros: velocidade de $40~m/min$,
distância da superfície de $230$ a $240~mm$ e ângulo da ferramenta de $30º$
a $90º$.
Cálculos de dimensionamento inicial alertaram sobre a flexibilidade considerável da
base, o que pode comprometer a qualidade do revestimento, ao alterar os
parâmetros controlados do processo.

O dimensionamento de uma base, ainda que pouco rígida, que cumpra com os
requisitos do sistema EMMA será o objeto central deste trabalho. Para isso,
propõe-se um método que, a partir de uma configuração de base, permita-se
quantificar o erro de trajetória e de velocidade e orientação do efetuador,
devido à flexibilidade desta base. Para validar o método, realiza-se um
experimento, por meio de instrumentação de uma base de testes, que está sendo
montada na empresa THIRTEEN ROBOTICS.

