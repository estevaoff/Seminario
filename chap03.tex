\chapter{Método Proposto}

\section{Manipulador}

A estrutura mecânica de um manipulador consiste numa sequência de
corpos rígidos (elos) e articulações (juntas); um manipulador é caracterizado
por um \textit{braço} que garante mobilidade, um \textit{pulso} que confere
destreza e um \textit{efetuador} que executa a tarefa requerida do robô
\cite{Siciliano2009}.
Neste trabalho, os elos e as juntas do manipulador serão considerados
perfeitamente rígidos. 

Ao se determinar uma tarefa, ou seja, uma trajetória do efetuador, o sistema
de controle do robô envia a informação de torques em cada junta a fim de
posicionar e orientar o efetuador ao longo da trajetória escolhida. Tais
torques resultam em um par de forças de ação e reação entre o
elo-inercial\footnote{Será definido como elo-inercial o primeiro elo, que
fixa o robô no ambiente, a fim de não causar confusão com a base estrutural
sobre a qual o robô está instalado.} do robô e a base sobre a qual este está fixado.
Os requisitos da tarefa, tais como trajetória, velocidade e aceleração do
efetuador; assim como os parâmetros do robô, como número de graus de
liberdade, alcance, limites angulares, velocidades de rotação e torques máximos
nas juntas, irão direcionar a escolha de manipuladores comerciais disponíveis
capazes de realizar a tarefa desejada. O manipulador, ao executar a
tarefa, é o responsável pelas forças e momentos externos que irão atuar sobre
sua base.

\section{Base do manipulador}

O método proposto requer que seja definida a base onde será instalado o
manipulador. 

Neste trabalho será considerado como base do manipulador todo o conjunto de
componentes que reage às cargas do manipulador, mas se deforma elasticamente.
E estão compreendidos entre o elo-inercial do manipulador e um corpo que
contém o referencial inercial do modelo.

\section{Modelo de base rígida}

Este modelo considera que o robô está instalado sobre uma base perfeitamente
rígida. Esta etapa permite a abordagem clássica de robótica para obter os
modelos de cinemática direta e cinemática inversa do manipulador.

\subsection{Cinemática direta}

A cinemática direta consiste em determinar a posição do efetuador como uma
função da configuração das juntas e elos do manipulador. Será utilizado o método
de Denavit-Hartenberg para o cálculo da equação de cinemética direta do
manipulador. Esta equação relaciona a posição e orientação do efetuador
em função das coordenadas generalizadas escolhidas.

Como resultado do estudo da cinemática direta, obtem-se a \textit{trajetória
ideal}, ou seja, aquela que o efetuador cumpre a tarefa sobre uma base
rígida.

\subsection{Cinemática inversa}

Na cinemtática inversa o objetivo é o contrário: dada uma configuração para
o efetuador, determina-se os parâmetros de junta que permitem tal configuração.
Logo, pode-se determinar, por exemplo, a trajetória desejada do efetuador e
obter como resultado os ângulos de cada junta que a solucionam permitindo o
cálculo dos torques necessários para realizar a tarefa sobre aquela trajetória.

Pela solução da cinemática inversa, os torques encontrados serão os dados de entrada
das forças externas que atuam em cada junta, posteriormente utilizados no modelo
de base flexível.

\section{Modelo de base flexível}

\subsection{Flexibilidade}
O objetivo principal deste trabalho baseia-se no modelo de uma base flexível
para o manipulador, ou seja, que se deforma elasticamente, desconsiderando
qualquer movimento de corpo rígido. Um corpo perfeitamente rígido é uma
idealização, geralmente assumida quando a deformação resultante do carregamento
é muito pequena. 

Neste trabalho, o interesse é quantificar a rigidez da base e avaliar os
casos em que este valor pode afetar a qualidade ou até a viabilidade de uma
determinada tarefa.

\subsection{CAD/CAE e MEF}
Ao se definir o projeto de uma base para o manipulador, a primeira etapa deste
modelo é definir um sistema de coordenadas conveniente e desenhar a
geometria que a compõe, em uma plataforma CAD/CAE (\textit{Computer Aided
Design}/\textit{Computer Aided Engineering}). 
A partir da geometria, são incluídas as propriedades dos materiais, restrições,
carregamentos e malha para uma simulação numérica pelo Método de Elementos Finitos (MEF). 

A base projetada para o sistema EMMA, consiste de 3 juntas: um trilho primário,
um trilho secundário, e uma base rotativa que forma um ângulo entre os 2
trilhos. Este sistema Prismático-Rotacional-Prismático (PRP), é responsável por
posicionar o robô próximo a pá, em determinado número de posições a fim de
cobrir toda a face da pá. A posição do robô pode ser definida, portanto, por um
conjunto de coordenadas generalizadas $\textbf{r} = \{ p1, \theta, p2 \}$, que
representam a posição no trilho primário, o ângulo da junta
rotativa e a posição no trilho secundário, respectivamente.

O modelo aplica, para cada simulação, uma força e um torque alinhados em cada
direção do sistema de referência da base. O deslocamento resultante, em cada
direção permite então obter uma Matriz de Rigidez $N\times N$ da estrutura, em
função de \textbf{r} onde $N$ é o número de graus de liberdade da estrutura,
avaliada na coordenada \textbf{r}, que representa o vetor posição do robô sobre
a base.

A partir de uma geometria complexa, a base é então reduzida a uma Matriz
de Rigidez, podendo ser representada por uma mola linear de 6 graus de liberdade
(3 translações e 3 rotações).

\subsection{Sophia-Maple e Equações de Kane}
\textit{Sophia} é um pacote de rotinas e
equações para auxiliar na descrição e solução especificamente de problemas
mecânicos.
Fornece um procedimento direto para se descrever matematicamente um
mecanismo e também se chegar rapidamente às equações de movimento do sistema.
Este pacote foi implementado no programa de álgebra simbólica
Maple\textsuperscript{\textregistered} pelo professor Martin
Lesser\footnote{Martin Lesser é professor de Mecânica do Royal Institute of
Technology, em Estocolmo na Suécia.}, descrevendo-o em seu livro intitulado
\textit{The Analysis of Complex Nonlinear Mechanical
Systems}\cite{lesser1995analysis}, e será utilizado no modelo do
sistema base-robô para se chegar às equações de movimento, pelo Método de
Kane. 

O método de Kane oferece as vantagens dos métodos de Newton-Euler e de
Lagrange, mas sem as desvantagens.
Com o uso de forças generalizadas, a necessidade de examinar forças de interação
restrição entre corpos é eliminada. Uma vez que o método de Kane não implica no
uso de funções energéticas, a diferenciação necessária para calcular velocidades
e acelerações podem ser obtidas através do uso de algoritmos baseados em
produtos vetoriais. O método de Kane fornece um meio elegante para desenvolver
as equações de dinâmica para sistemas multicorpos que se prestam à computação
numérica automatizada.
(\citet{huston1991multibody})

Apesar de ter uma formulação mais complexa que os
outros métodos, este pode ser melhor sistematizado para dinâmica de
multicorpos, utilizando as rotinas do pacote de subrotinas
especificamente desenvolvido por \citet{lesser1995analysis} para
essa finalidade, Sophia-Maple.

Os torques obtidos na análise do modelo de base rígida são fornecidos para
resolver o sistema de equações de movimento. Devido à complexidade do sistema de
equações e não-linearidades, a solução será dada por método numérico.

O efeito dinâmico da base, modelada como uma mola de 6 graus de liberdade, irá
revelar a diferença de trajetória entre os dois modelos de base e permitirá
quantificar essa diferença, em termos de posições, orientações e velocidades do
efetuador.

\section{Validação Experimental}

A fim de se buscar uma validação experimental do método proposto, será realizado
um experimento.
Acelerômetros fixados na placa-base do robô permitirão avaliar o comportamento
dinâmico da base. O experimento consistirá em definir uma trajetória do
efetuador, idêntica a do modelo teórico e coletar as acelerações de um ponto
da base no sistema de referência. 

Um modelo de base flexível correspondente deve ser realizado, e como resultado,
obter as acelerações no mesmo ponto analisado no modelo físico. A comparação
dos resultado teóricos e experimentais permitirá a discussão sobre a validação
do método proposto.

